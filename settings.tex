% 导言区
\usepackage{graphicx}  % 用于处理图形和插图
\usepackage{algorithm} % 算法
\usepackage{algpseudocode} % 伪代码
\usepackage{amsmath}   % 提供扩展的数学公式排版功能
\usepackage{amssymb}   % 提供扩展的数学公式排版功能
\usepackage{amsthm}    % 提供扩展的数学定理证明功能
\usepackage{mathrsfs}  % 花体字母
\usepackage{diagbox}   % 提供表格对角线方框
\usepackage{tabularx}  % 提供表格排版扩展功能
\usepackage{float}     % 强制表格放置到指定位置
\usepackage{array}     % 提供表格排版扩展功能
\usepackage{booktabs}  % 用于设置三线表
\usepackage{hyperref}  % 创建可点击的超链接
\usepackage{titlesec}  % 用于设置标题样式
\usepackage{minted}    % 用于代码高亮
\usepackage{enumitem}  % 用于自定义列表环境
\usepackage{todonotes} % 用于编写当前文档的todo
\usepackage{xargs}     % 用于自定义命令
\usepackage{cleveref}  % 用于交叉引用
\usepackage{glossaries-cn}                % 用于生成术语表
\usepackage[no-math]{fontspec}            % 用于设置字体
\usepackage[table, dvipsnames]{xcolor}    % 提供颜色处理功能
\usepackage[a4paper, left=3cm, right=3cm, top=3cm, bottom=3cm]{geometry}					  % 更改边栏

% 定义todo命令
\setlength{\marginparwidth}{2.25cm}
\newcommand{\unsure}[2][1=]{\todo[linecolor=red,backgroundcolor=red!25,bordercolor=red]{#2}}
\newcommand{\change}[2][1=]{\todo[linecolor=blue,backgroundcolor=blue!25,bordercolor=blue]{#2}}
\newcommand{\info}[2][1=]{\todo[linecolor=OliveGreen,backgroundcolor=OliveGreen!25,bordercolor=OliveGreen]{#2}}
\newcommand{\improvement}[2][1=]{\todo[linecolor=Plum,backgroundcolor=Plum!25,bordercolor=Plum]{#2}}

% 微分符号
\newcommand*{\dif}{\mathop{}\!\mathrm{d}}
% 定义、定理、引理、推论、命题、公理、性质
\newtheorem{definition}{定义}[chapter]  % 定义
\newtheorem{theorem}{定理}[chapter]     % 定理
\newtheorem{lemma}{引理}[chapter]       % 引理
\newtheorem{corollary}{推论}[chapter]   % 推论
\newtheorem{proposition}{命题}[chapter] % 命题
\newtheorem{axiom}{公理}                % 公理
\newtheorem*{property}{性质}            % 性质
% 自定义cref环境名称为中文
\newcounter{inequality}
\newenvironment{inequality}
	{\refstepcounter{inequality}\begin{equation}}
	{\end{equation}}
\newenvironment{inequality*}
	{\refstepcounter{inequality}\begin{equation*}}
	{\end{equation*}}
\crefname{inequality}{不等式}{不等式}
\crefname{equation}{公式}{公式}
\crefname{definition}{定义}{定义}
\crefname{theorem}{定理}{定理}
\crefname{lemma}{引理}{引理}
\crefname{corollary}{推论}{推论}
\crefname{proposition}{命题}{命题}  
\crefname{axiom}{公理}{公理} 
\crefname{property}{性质}{性质}

% 设置英文和中文字体
\setmainfont{Times New Roman} % 设置英文主字体
\setCJKmainfont{SimSun}[BoldFont=SimHei, ItalicFont=KaiTi] % 设置中文主字体

% 设置标题字体
\titleformat{\chapter}[display]
{\normalfont\huge\bfseries\heiti}
{\chaptername\ \thechapter}
{20pt}{\huge}
[\vspace{10pt}\titlerule] % 添加标题下划线
\titleformat{\section}
{\normalfont\Large\bfseries\heiti}
{\thesection}
{1em}
{}
\titleformat{\subsection}
{\normalfont\large\bfseries\heiti}
{\thesubsection}
{1em}
{}
\titleformat{\subsubsection}
{\normalfont\normalsize\bfseries\heiti\color{teal}}
{\thesubsubsection}
{1em}
{}
\titleformat{\paragraph}[runin]
{\normalfont\normalsize\bfseries\heiti}
{}
{0em}
{}
[:]
\titleformat{\subparagraph}[runin]
{\normalfont\normalsize\bfseries\heiti}
{}
{0em}
{}

% 设置超链接颜色
\hypersetup{
	colorlinks=true,   % 启用颜色链接
	linkcolor=blue,    % 链接的颜色
	citecolor=green,   % 引用链接的颜色
	urlcolor=blue      % URL链接的颜色
}

% 定义封面标题页
\title{高等代数专题}
\author{倪兴程 \thanks{Email: 19975022383@163.com}}
\date{2025年2月20日}
